In the following we will always consider an SDE of the form
\begin{align}
\dd X_t = a(X_t)\dd t + b(X_t) \dd W_t, \quad t\in[0,T]
\end{align}
Corresponding Euler approximation:
\begin{align}
Y_{\tau_{n+1}}&= Y_{\tau_{n}}+a(Y_{\tau_{n}})(\tau_{n+1}-\tau_{n})+b(Y_{\tau_{n}})(W_{\tau_{n+1}}-W_{\tau_{n}}) \label{eq:Euler_approx}\\
\text{with}\qquad0 &= \tau_{0}<\tau_{1}<...<\tau_{n}<...<\tau_{N} = T
\end{align}
Lipschitz condition: $\exists K>0$ such that $\forall x,y \in \R$: 
\begin{align}
|a(x)-a(y)|+|b(x)-b(y)|\leq K|x-y|\label{eq:Lip_cond}
\end{align}
Numerical stability: Let $Y$ be a discrete time approximation with $Y_0$ at time $t_0$ and $\delta$ the maximum time step. Let $\bar{Y}$ be an approximation corresponding to the same time discretisation but starting at a different value $\bar{Y}_0$ at time $t_0$. Then $Y$ is numerically stable If for any finite  time interval $[t_0,T]$ there exists a positive constant $\Delta_0$ such that for each $\epsilon>0$ and each $\delta \in (0,\Delta_0)$
\begin{align}
\lim_{|Y_0-\bar{Y}_0|\rightarrow 0}\sup_{t\in[t_0,T]} P\left(|Y_{n_t}-\bar{Y}_{n_t}|\geq \epsilon\right) =0
\end{align}
\textbf{Proposition}: \\
If an SDE fulfils the Lipschitz condition and its initial value $X_0$ is $\mathscr{F}_0$-measurable with $\E[X_0^2]<\infty$, the Euler approximation is numerical stable.\\ \\
\textbf{Proof}:\\
We will start by putting an upper bound on $\left|Y_{n+1}-\bar{Y}_{n+1}\right|$
\begin{align}
\left|Y_{n+1}-\bar{Y}_{n+1}\right| = \left|Y_n-\bar{Y}_{n}+[a(Y_n)-a(\bar{Y}_n)](\tau_{n+1}-\tau_n) +[b(Y_n)-b(\bar{Y}_n)](W_{n+1}-W_n)\right|
\end{align}
We will bound this from above, by using the triangular inequality and the fact that $\tau_{n+1}-\tau_n\leq \delta$
\begin{align}
\left|Y_{n+1}-\bar{Y}_{n+1}\right| \leq  \left|Y_n-\bar{Y}_{n}\right|+\delta\left|a(Y_n)-a(\bar{Y}_n)\right| +\left|\Delta W_{n+1}\right|\left|b(Y_n)-b(\bar{Y}_n)\right|
\end{align}
with $\Delta W_{n+1}:=(W_{n+1}-W_n)$. With the Lipschitz condition of \cref{eq:Lip_cond} we know that surely \mbox{$\left|a(Y_n)-a(\bar{Y}_n)\right|\leq K \left|Y_n-\bar{Y}_{n}\right|$} and \mbox{$\left|b(Y_n)-b(\bar{Y}_n)\right|\leq K \left|Y_n-\bar{Y}_{n}\right|$} and thus
\begin{align}
\left|Y_{n+1}-\bar{Y}_{n+1}\right|\leq \left|Y_n-\bar{Y}_{n}\right|(1+\delta K+|\Delta W_{n+1}|K)
\end{align}
We can use this result recursively to obtain
\begin{align}
\left|Y_{n+1}-\bar{Y}_{n+1}\right|\leq \left|Y_0-\bar{Y}_{0}\right|\prod_{i=1}^{n+1}\underbrace{(1+\delta K+|\Delta W_{i}|K)}_{:=Z_i}
\end{align}
Where we defined the new sequence of random variables $Z_i :=(1+\delta K+|\Delta W_{i}|K) $. If one random variable is always equal or greater than the other, then its probability to being greater than $\epsilon$ is surely equally or greater as well, and thus we can write
\begin{align}
P(\left|Y_{n}-\bar{Y}_{n}\right|\geq \epsilon)&\leq P\left(\left|Y_0-\bar{Y}_0\right|\prod_{i=1}^n Z_i \geq \epsilon\right)\\
& = P\left(\prod_{i=1}^n Z_i \geq \epsilon/\left|Y_0-\bar{Y}_0\right|\right)\\
& = 1 - P\left(\prod_{i=1}^n Z_i \leq \epsilon/\left|Y_0-\bar{Y}_0\right|\right)\label{eq:a_gegen_wahrsch}
\end{align}
Since the the Wiener increments are independent, we know, that the $Z_i$ are independent and thus can write
\begin{align}
P\left(\prod_{i=1}^n Z_i \leq \epsilon/\left|Y_0-\bar{Y}_0\right|\right) = P_1(A_1)\manmult P_n(A_n), \\
A_i = \{z_i\in\R:z_1\manmult z_n \geq \epsilon/\left|Y_0-\bar{Y}_0\right|\}\, \forall i \in \{1,...,n\}
\end{align}
We can make another approximation by reducing the sets $A_i$ such that all random variables fulfil $z_i\leq (\epsilon/\left|Y_0-\bar{Y}_0\right|)^{1/n}$, so 
\begin{align}
P\left(\prod_{i=1}^n Z_i \leq \epsilon/\left|Y_0-\bar{Y}_0\right|\right)  \geq \prod_{i=1}^n P_i\left(Z_i\leq \sqrt[n]{ \epsilon/\left|Y_0-\bar{Y}_0\right|}\right)
\end{align}
Putting this into \cref{eq:a_gegen_wahrsch} gives
\begin{align}
P(\left|Y_{n}-\bar{Y}_{n}\right|\geq \epsilon)&\leq 1 - \prod_{i=1}^n P_i\left(Z_i\leq \sqrt[n]{ \epsilon/\left|Y_0-\bar{Y}_0\right|}\right) \label{eq:a_absch_gegen_wahrsch}
\end{align}
The probability density of a Wiener increment is given by $\Delta_{\tau_n} - \Delta_\tau \sim  \mathcal{N}(0,\tau_{n+1}-\tau_n)$. From this we can conclude that the probability density of the $Z_i$ is given by 
\begin{align}
Z_i \sim \begin{cases} 
      0 & Z_i< 1+\delta K \\
      2 \mathcal{N}(1+\delta K,\sqrt{K} \delta_i) &  Z_i\geq 1+\delta K
   \end{cases}
\end{align}
with $\delta_i = \tau_{i+1}-\tau_i$. So we can write
\begin{align}
P_i\left(Z_i\leq \sqrt[n]{ \epsilon/\left|Y_0-\bar{Y}_0\right|}\right) =
 \begin{cases} 
      0 & a< 1+\delta K \\
      \int_{1+\delta K}^{a} 2 \mathcal{N}(1+\delta K,\sqrt{K} \delta_i) \dd z_i & a\geq 1+\delta K
   \end{cases}
\end{align}
with $a=\sqrt[n]{\epsilon/\left|Y_0-\bar{Y}_0\right| }$.
Since $0\leq P(Z_i>\tilde{\epsilon})\leq 1$ for any $\tilde{\epsilon}$, we know together with \cref{eq:a_absch_gegen_wahrsch}, that the supremum is bounded by
\begin{align}
1 - \prod_{i=1}^N P_i\left(Z_i\leq \sqrt[N]{ \epsilon/\left|Y_0-\bar{Y}_0\right|}\right) &= \sup_{t\in[t_0,T]}  1 - \prod_{i=1}^{n_t} P_i\left(Z_i\leq \sqrt[n_t]{ \epsilon/\left|Y_0-\bar{Y}_0\right|}\right) \\ &\geq\sup_{t\in[t_0,T]} P\left(|Y_{n_t}-\bar{Y}_{n_t}|\geq \epsilon\right)
\end{align}
If we now consider the limit $\lim_{\left|Y_0-\bar{Y}_0\right| \rightarrow 0}$, which is equivalent to $\lim_{a\rightarrow \infty}$ for finite $\epsilon$, we can deduce that
\begin{align}
\lim_{a\rightarrow \infty} P_i(Z_i\leq a) &= \lim_{a\rightarrow \infty} \int_{1+\delta K}^{a} 2\mathcal{N}(1+\delta K,\sqrt{K}\delta_i)\dd z_i \\
&= \int_{1+\delta K}^{\infty} 2\mathcal{N}(1+\delta K,\sqrt{K}\delta_i)\dd z_i =1\label{eq:a_int_is_1}
\end{align}
So we can finally write with the uniqueness of limits
\begin{align}
\lim_{\left|Y_0-\bar{Y}_0\right| \rightarrow 0} \sup_{t\in[t_0,T]} P\left(|Y_{n_t}-\bar{Y}_{n_t}|\geq \epsilon\right) &\leq \lim_{a\rightarrow \infty} 1 -  \prod_{i=1}^N P_i\left(Z_i\leq \sqrt[N]{ \epsilon/\left|Y_0-\bar{Y}_0\right|}\right)\\
&=1 -  \prod_{i=1}^N \underbrace{\lim_{a\rightarrow \infty} P_i\left(Z_i\leq \sqrt[N]{ \epsilon/\left|Y_0-\bar{Y}_0\right|}\right)}_{=1,\,\text{\cref{{eq:a_int_is_1}}}} = 0
\end{align}
Where we were allowed to interchange the product and limit since the factors have each a finite limit.
And since probabilities can not be negative we arrive at
\begin{align}
\lim_{\left|Y_0-\bar{Y}_0\right| \rightarrow 0} \sup_{t\in[t_0,T]} P\left(|Y_{n_t}-\bar{Y}_{n_t}|\geq \epsilon\right) = 0,
\end{align}
which concludes the poof.